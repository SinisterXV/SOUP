\chapter{Conclusions}\label{Conclusions}

Several possible improvements can be made to our microprocessor, as we did not have the chance to fully develop all the feasible features of a RISC processor. 

Pipeline optimisation could be a starting point: we did not design units to handle data and control hazards  as we handled the issue via software. 
However, this is inefficient and proper hardware components should be added: they include a forwarding unit, a stall controller, and a pipeline flush controller.
They would greatly help to improve the overall performance and usability of the device. 
Additional upgrades would include adding a Branch History Table or a Branch Target Buffer to perform branch prediction and reduce branch/jump penalties.

Another desirable feature would be floating point execution units, as currently only integer arithmetic is supported.

Optimisation of the critical path is a very important point, since at the moment it's 6.5 \si{\nano\second} due to the 64-bit ripple carry adder inside the multiplier. 
Possible solutions include: 
\begin{itemize}
    \item using a 64-bit P4 adder;
    \item implementing a 32-bit multiplier instead of a 64-bit one;
    \item pipelining the multiplier, increasing its throughput.
\end{itemize}

One more improvement can be implementing power optimisation techniques, such as clock gating, to reduce the power consumption.

In conclusion, the project has been a challenging yet rewarding journey for the team. 
The effort has been remarkable, as we delved into the intricacies of microprocessor architecture and digital design. 
Through this project, we acquired skills in complex functionality integration and the implementation of advanced algorithms like the SRT division.
We have refined our abilities in problem-solving, critical thinking, and collaboration, navigating through a high-complexity design and overcoming obstacles. 
This project not only provided a platform for applying theoretical knowledge but also fostered a deeper understanding of the intricacies involved in developing complex features. 
The experience gained throughout this project sets a foundation for further exploration in the field of microprocessor design and development.





