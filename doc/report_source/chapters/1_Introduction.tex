\chapter{Introduction}\label{Introduction}

The implementation of a pipelined RISC microprocessor with advanced functionalities is a remarkable feat in the realm of digital design and computer architecture. 
Our implementation of the DLX microprocessor, made of a controller and a datapath, includes sophisticated components such as a radix-4 Booth multiplier and a radix-4 SRT divider.

The DLX architecture, inspired by the MIPS, is the foundation of our implementation. 
Its simple instruction set enables efficient execution of complex programs, making it an ideal choice for a wide range of applications. 
However, the addition of advanced functionalities takes the DLX microprocessor to new heights, making it capable of handling computationally intensive tasks. 

Our processor is a combination of a hardwired control unit and a datapath, working together to execute instructions and perform complex computations. 
On one hand, the control unit is responsible for generating the control signals that coordinate the operations within the datapath. 
On the other hand, the datapath is the core processing unit, consisting of arithmetic/logic units, registers and multiplexers. 
It carries out the actual calculations and manipulations of data under the influence of the control unit. 
The detailed explanation of the implementation of these components can be found in \autoref{controller_ch} and \autoref{datapathch}.

As for the pipeline, a static one has been implemented and consists of five stages: fetch, decode, execute, memory and write-back. 
By breaking down each instruction's execution into multiple stages, the pipelined datapath enables the concurrent execution of several instructions in different stages, resulting in high throughput.

The inclusion of a radix-4 SRT divider adds to the microprocessor's capabilities. 
The SRT division algorithm utilises a combination of shifts, additions and subtractions to perform division operations with exceptional speed and precision. 
We carefully studied and collected the theory behind it and the results of our work can be found in  \autoref{Chapter_SRT}.
The employment of a radix-4 algorithm further increases the division efficiency by reducing the number of required iterations.
The detailed explanation of the implementation can be found in \autoref{Chapter_Impl_Div}.

The radix-4 Booth multiplier integrated into the microprocessor is another noteworthy feature. 
By employing a combination of shifts and additions/subtractions, the multiplier reduces the number of required intermediate products and minimises the overall latency, quickly computing long multiplications. 
Its implementation is discussed in  \autoref{Chpater_Impl_mul}. 

An additional feature of our implementation is the inclusion of automated simulation and synthesis scripts, which streamline the design process. 
These scripts make it easier to verify and validate the processor's functionality and performance. 
This automation significantly reduces the manual effort required for simulation and synthesis tasks, enabling  the user to focus more on design refinement and optimisation. 
Further details about the simulation and synthesis automation can be found in \autoref{sim_chap} and \autoref{syn_chap}, respectively.